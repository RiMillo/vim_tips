%! TEX root = vim_tips.tex

%% vim tips and stuff.
%%
%% Copyright 2021 Riccardo Milani
%%
%% Licensed under the "THE BEER-WARE LICENSE" (Revision 42):
%% Riccardo Milani wrote this file. As long as you retain this notice you
%% can do whatever you want with this stuff. If we meet some day, and you think
%% this stuff is worth it, you can buy me a beer or coffee in return


% Authors: Riccardo Milani, X12

\documentclass[a4paper,12pt,%
              final%
              %draft%
              ]{article}

\usepackage[english]{babel}
\usepackage[utf8]{inputenc}
\usepackage[T1]{fontenc}
%\usepackage{lmodern}

\usepackage[top=3cm, bottom=3cm, left=2cm, right=2cm]{geometry}

\usepackage{xcolor}
\definecolor{BlueX}{RGB}{0,62,92}
\def\maincolor{BlueX}
\definecolor{mLightBrown}{HTML}{EB811B}

\usepackage{verbatim}
\usepackage{fancyvrb}
\usepackage{minted} % It supports vimscript differently than listing

\usepackage{titlesec} % Modify the style of the sections, chapters...
\titleformat{\section} %command
    %[display] %shape
    {\color{\maincolor}\Large\bfseries\itshape} %format
    {\color{\maincolor}\Large\bfseries\itshape\thesection~-} %label
    {1ex} %sep
    {} %before-code
    %{} %after-code
\titleformat{\subsection} %command
    %[display] %shape
    {\color{\maincolor}\large\bfseries} %format
    {\color{\maincolor}\large\bfseries\thesubsection~-} %label
    {.5ex} %sep
    {} %before-code
    %{} %after-code

\usepackage[%
  pdfpagelabels,
  urlbordercolor=\maincolor,
  linkbordercolor=mLightBrown,
  pdfborderstyle={/S/U/W 1}
]{hyperref}
\newcommand*{\fullref}[1]{\hyperref[{#1}]{\autoref*{#1} - \textit{\nameref*{#1}}}}

\usepackage[numbered]{bookmark}

\parindent=0pt
\setlength{\parskip}{2ex}

\title{\color{\maincolor}\Huge\bfseries\scshape Random vim tips}
\author{\vspace{-7ex}}
\date{\vspace{-7ex}}
\addto\captionsenglish{% Needed if babel is used. It is needed for every language that will be used.
  \renewcommand{\abstractname}{\vspace*{-\baselineskip}}%
}

\newcommand{\vim}{\texttt{vim}}
\newcommand{\gvim}{\texttt{gvim}}
\newcommand{\vimscript}{\texttt{Vimscript}}
%
\begin{document}
%
%
\maketitle
%
\vspace*{-2.5cm}

\pdfbookmark[1]{Abstract}{abstract}
\begin{abstract}
\parindent=0pt
\setlength{\parskip}{2pt}
\noindent
I collect here tips\&tricks about \texttt{[g]vim}. This is not a tutorial: I won't
teach you
\href{https://stackoverflow.com/questions/11828270/how-do-i-exit-the-vim-editor}{how
to exit the \vim{} editor}. It's just some useful stuff that I do not seem to
be able to remember.

I gathered this over the years with some Googlin' and with thousands of visits on
StackOverflow and \href{https://vim.fandom.com/wiki/Vim_Tips_Wiki}{Vim Famdom}, to
which you are referred if you don't find what you need here (this will very often be
the case), and sometimes even \href{https://www.reddit.com/r/vim/}{Reddit}. You may
also check out \href{https://vimawesome.com/}{vimawesome} which provides a nice lists
of useful of plugins.

Some notation. I'll use classical stuff. \texttt{<C-a>} is control and A keys.
\texttt{<S-a>} is shift and A keys. \texttt{<CR>} is the Carriage Return, aka Enter.
You get the idea.

So far, very few tips have been collected and there is no special structure, just
some bullet points. Things will get better (I hope).

I am no \vim{}-wiz: I sure made some mistakes here and there and I won't be
surprised if I'm selling \gvim{} stuff as it were available in terminal
\texttt{vim}, too. Do not hesitate to hit me up for fixes or adds!
\end{abstract}
%
\section{Generic stuff}
\begin{itemize}
  \item Write to and read from file
    \begin{itemize}
      \item Write: from manual (almost)
\begin{minted}{vim}
:[range]w[rite][!] [++opt]  {file}
      Write the specified lines to {file}, unless it already exists
      and the 'writeany' option is off. '!' overwrites
:[range]w[rite][!] [++opt] >> {file}
      Append the specified lines to {file}. '!' forces the write even
      if file does not exist.
\end{minted}
      \item Read: from manual
\begin{minted}{vim}
:{range}r[ead] [++opt] [name]
      Insert the file [name] (default: current file) below
      the specified line.

:[range]r[ead] [++opt] !{cmd}
      Execute {cmd} and insert its standard output below
      the cursor or the specified line. A temporary file is
      used to store the output of the command which is then
      read into the buffer. 'shellredir' is used to save
      the output of the command, which can be set to include
      stderr or not. {cmd} is executed like with ":!{cmd}",
      any '!' is replaced with the previous command |:!|.
\end{minted}
    \end{itemize}
  \item Get current value of\ldots
    \begin{itemize}
      \item an option \verb|opt|: \verb|:set <opt>?|
      \item a variable \verb|var|: \verb|:echo <var>| (for some native variables one should use \verb|expand()|, as in \verb|expand("%")|)
    \end{itemize}
  \item Open stuff under the cursor. In normal mode:
    \begin{itemize}
      \item \texttt{gf}: open file under cursor. \texttt{<C-w>f}: split the window
        and open file under cursor.
      \item \texttt{gx}: open URL / http under cursor.
    \end{itemize}
  \item Manipulating filnenames and paths:
    \begin{itemize}
      \item \verb|%|: wildcard for current file;
      \item \verb|<|: current file without extension;
      \item \verb|%:p[:h]|: complete path of current file with [without] filename
    \end{itemize}
  \item Variables
    \begin{itemize}
      \item Concatenate strings: use a dot "\texttt{.}", as in: \verb|"A B" . " C D"|
    \end{itemize}
  \item Edit remote file: use the \texttt{scp} protocol. For instance (notice the
    double-slashes)
\begin{verbatim}
vim scp://user@domain//path/on/the/remote/machine
\end{verbatim}
  \item Motions
    \begin{itemize}
      \item \texttt{j} and \texttt{k}: up and down line-wise. \texttt{h} and
        \texttt{l}: left and right character-wise (of course the arrows work as
        well). Notice that it is one true line, and not wrapped-lines: in order to
        move from one wrapped-line to another, use \texttt{gj} or \texttt{gk}.
      \item The same reasoning applies to windows, so, for instance, \verb|<C-W>k|
        moves you to the above window
      \item \texttt{w} move to the beginning of the following word: notice that a
        word is usually delimited by a space or a punctuation point (or similar). The
        capital version, \texttt{W}, operates on block words, basically the words are
        delimited only by spaces. \texttt{e} and \texttt{E}, same as before but at
        the \emph{end} of the following word. \texttt{b} and \texttt{B} same as
        before but of the \emph{previous} word.
      \item \texttt{]]}: go the beginning of the next section / function. \texttt{[[}
        same but previous section / function
      \item \verb|{| or \verb|}|: up or down one paragraph (blank line)
      \item \verb|(| or \verb|)|: up or down one sentence (dot or blank line)
      \item \verb|%|: move to the other end of parenthesis or environment in normal / visual mode
      \item \verb|o|: move to the other end of the selected area in visual mode
    \end{itemize}
  \item \href{https://vim.fandom.com/wiki/Find_in_files_within_Vim}{Grep \& vimgrep}
\begin{verbatim}
:[vim]grep /pattern/[g][j] file.txt
\end{verbatim}
    \begin{itemize}
      \item \texttt{vim[grep]} uses a internal algorithm of \texttt{vim}.
        \texttt{grep} uses an external program.
      \item \texttt{g}: \emph{all} matches are shown (instead of only one per line).
      \item \texttt{j}: do NOT jump to first match.
      \item \texttt{[vim]grep} store the results in the \emph{quickwindow}.
        \texttt{l[vim]grep} in the \emph{location list}.
    \end{itemize}
  \item Text objects: They can be used with commands like \texttt{c}, \texttt{d} or
    \texttt{v}.
    \begin{itemize}
      \item \texttt{w} word, \texttt{s} sentence, \texttt{p} paragraph, \texttt{B}
        block (such as braces for \texttt{C} language), and delimiters such as
        parentheses \verb|)]}| or quotes \verb|"'`|
      \item Modifiers: \texttt{i} internal or \texttt{a} whole argument. For
        instance, \texttt{vi)} will select all the text \emph{inside} the current
        parentheses but not characters \texttt{(} and \texttt{)}, whereas
        \texttt{va)} will select all the text and the parentheses as well.
    \end{itemize}
  \item Registers: the bases \href{https://www.brianstorti.com/vim-registers/}{here}.
    \vim{} stores stuff that you copy or delete in registers. They are accessible
    with ``\texttt{"}'' (quotes) plus the register name. Several types of register
    exist:
    \begin{itemize}
      \item Unnamed register: stores the content of the last delete-like or yank
        command
      \item Numbered registers, from 0 to 9: store the content of the recent
        delete-like or yank commands in anti-chronological order. \texttt{"1} is the
        last, \texttt{"2} second-to-last, etc\ldots \texttt{"0} is the last yank
      \item literal registers, \texttt{a} to \texttt{z} (case insensitive): filled by
        the users
      \item Read-only registers:
        \begin{itemize}
          \item ``\texttt{.}'': last inserted text
          \item ``\verb!%!'': name of the current file
          \item ``\texttt{:}'': most recent executed command
        \end{itemize}
      \item Last-searched register, ``/''
      \item GUI-related registers:
        \begin{itemize}
          \item ``*'': selected with the mouse
          \item ``+'': clipboard
          \item ``\verb|~|'': last drag-n-drop
        \end{itemize}
      \item Small-delete register, ``-'': last command that delete less than a line
      \item Black hole register, ``\verb|_|'': discards everything
      \item See the content of the registers with \verb|:reg[isters]|
    \end{itemize}
  \item Recordings or
    \href{https://vim.fandom.com/wiki/Macros#Viewing_a_macro}{macros}:
    \begin{itemize}
      \item \verb|q{0-9a-zA-Z"}|: record typed characters into register
        \verb|{0-9a-zA-Z"}| (uppercase to append). Stop recording with \verb|q|.
      \item \verb|@{0-9a-zA-Z"}|: execute the content of register
        \verb|{0-9a-z".=*+}|. It accepts count
    \end{itemize}
  \item Sort lines: \verb|:<range>sort|, see \href{https://vim.fandom.com/wiki/Sort_lines}{here}
    \begin{itemize}
      \item \verb|:sort!|: reverse order
      \item \verb|:sort n|: use numeric sort
      \item \verb|:sort u|: delete duplicated lines (mnemonic: unique)
    \end{itemize}
  \item File names: have a look \href{https://vim.fandom.com/wiki/Get_the_name_of_the_current_file}{here}.
    Register \verb|%| contains the name of the current file, \verb|#| the name of alternative file.
    Modifiers (see \verb|:he filename-modifiers|):
    \begin{itemize}
      \item \verb|@%| (at): directory and name of file w.r.t.~current working directory
      \item \verb|%:t| (tail): name of file only
      \item \verb|%:p| (path): full path
      \item \verb|%:h| (head): path without file name
      \item \verb|%:r| (root): name of file without extension
      \item \verb|%:e| (extension): extension
      \item Combinations are accepted: \verb|%:p:h|: path to the directory containing the current file
    \end{itemize}
  \item Regex
    \begin{itemize}
      \item \href{http://vimregex.com/#Non-Greedy}{Non-greedy}: \verb|\{-}| as few as possible
      \item Not: atom \verb|\@!|, have a look \href{https://vim.fandom.com/wiki/Search_for_lines_not_containing_pattern_and_other_helpful_searches}{here}
        \begin{itemize}
          \item Find all \texttt{foo} not followed by \texttt{bar}: \verb|/foo\(bar\)\@!| or \verb|/\(foo\)\@<!bar|
        \end{itemize}
    \end{itemize}
\end{itemize}
%
\section{\vimscript{}'ing}
\vim{} has its own language for developing plugins and extend the editor as one wishes, it is called \vimscript{}, or, less often, \texttt{VimL}.
\begin{itemize}
  \item Tutorial with examples: \href{https://learnvimscriptthehardway.stevelosh.com/}{here}
  \item Variables: some prefixes are usually added to the name of the variable and this lets you choose the scope of the variable, cf \href{https://developer.ibm.com/articles/l-vim-script-1/#values-and-variables}{here}
    \begin{itemize}
      \item \texttt{g:foo} global
      \item \texttt{s:foo} local to the current script file
      \item \texttt{w:foo} local to the current editor window
      \item \texttt{t:foo} local to the current editor tab
      \item \texttt{b:foo} local to the current editor buffer
      \item \texttt{l:foo} local to the current function
      \item \texttt{a:foo} parameter of the current function
      \item \texttt{v:foo} one the variables that \vim{} predefines
    \end{itemize}
\end{itemize}

\section{Interesting plugins}
\begin{itemize}
  \item \href{https://github.com/puremourning/vimspector}{\texttt{vimspector}}: a
    debugger in \vim{}
  \item \href{https://github.com/chrisbra/csv.vim}{\texttt{csv}}: well, work with \texttt{csv files}
  \item \href{https://github.com/jalvesaq/Nvim-R}{\texttt{Nvim-R}}: working with \texttt{R}
  \item \href{https://github.com/lervag/vimtex}{\texttt{vimtex}}: working with \LaTeX{}
\end{itemize}
%
%
\end{document}
