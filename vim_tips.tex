%! TEX root = vim_tips.tex

%% vim tips and stuff.
%%
%% Copyright 2021 Riccardo Milani
%%
%% Licensed under the "THE BEER-WARE LICENSE" (Revision 42):
%% Riccardo Milani wrote this file. As long as you retain this notice you
%% can do whatever you want with this stuff. If we meet some day, and you think
%% this stuff is worth it, you can buy me a beer or coffee in return


% Authors: Riccardo Milani, X12

\documentclass[a4paper,12pt,%
              final%
              %draft%
              ]{article}

\usepackage[english]{babel}
\usepackage[utf8]{inputenc}
\usepackage[T1]{fontenc}
%\usepackage{lmodern}

\usepackage[top=3cm, bottom=3cm, left=2cm, right=2cm]{geometry}

\usepackage{xcolor}
\definecolor{BlueX}{RGB}{0,62,92}
\def\maincolor{BlueX}
\definecolor{mLightBrown}{HTML}{EB811B}

\usepackage{verbatim}
\usepackage{fancyvrb}

\usepackage{titlesec} % Modify the style of the sections, chapters...
\titleformat{\section} %command
    %[display] %shape
    {\color{\maincolor}\Large\bfseries\itshape} %format
    {\color{\maincolor}\Large\bfseries\itshape\thesection~-} %label
    {1ex} %sep
    {} %before-code
    %{} %after-code
\titleformat{\subsection} %command
    %[display] %shape
    {\color{\maincolor}\large\bfseries} %format
    {\color{\maincolor}\large\bfseries\thesubsection~-} %label
    {.5ex} %sep
    {} %before-code
    %{} %after-code

\usepackage[%
  pdfpagelabels,
  urlbordercolor=\maincolor,
  linkbordercolor=mLightBrown,
  pdfborderstyle={/S/U/W 1}
]{hyperref}
\newcommand*{\fullref}[1]{\hyperref[{#1}]{\autoref*{#1} - \textit{\nameref*{#1}}}}

\usepackage[numbered]{bookmark}

\parindent=0pt
\setlength{\parskip}{2ex}

\title{\color{\maincolor}\Huge\bfseries\scshape Random vim tips}
\author{\vspace{-7ex}}
\date{\vspace{-7ex}}
\addto\captionsenglish{% Needed if babel is used. It is needed for every language that will be used.
  \renewcommand{\abstractname}{\vspace*{-\baselineskip}}%
}
%
\begin{document}
%
%
\maketitle
%
\vspace*{-2.5cm}

\pdfbookmark[1]{Abstract}{abstract}
\begin{abstract}
\parindent=0pt
\setlength{\parskip}{2pt}
\noindent
I collect here tips\&tricks about \texttt{[g]vim}. This is not a tutorial: I won't
teach you
\href{https://stackoverflow.com/questions/11828270/how-do-i-exit-the-vim-editor}{how
to exit the \texttt{vim} editor}. It's just some useful stuff that I seem not to be
able to remember.

I gathered this over the years with some Googlin' and with thousands of visits on
StackOverflow and \href{https://vim.fandom.com/wiki/Vim_Tips_Wiki}{Vim Famdom}, to
which you are referred if you don't find what you need here (this will very often be
the case), and sometimes even Reddit. You may also check out
\href{https://vimawesome.com/}{vimawesome} which provides a nice lists of useful of
plugins.

Some notation. I'll use classical stuff. \texttt{<C-a>} is control and A keys.
\texttt{<S-a>} is shift and A keys. \texttt{<CR>} is the Carriage Return, aka Enter.
You get the idea.

So far, very few tips have been collected and there is no special structure, just
some bullet points. Things will get better (I hope).

I am no \texttt{vim}-wiz: I sure made some mistakes here and there and I won't be
surprised if I'm selling \texttt{gvim} stuff as it were available in terminal
\texttt{vim}, too. Do no not hesitate to hit me up for fixes or adds!
\end{abstract}
%
\begin{itemize}
  \item Open stuff under the cursor. In normal mode:
    \begin{itemize}
      \item \texttt{gf}: open file under cursor. \texttt{<C-w>f}: split the window
        and open file under cursor.
      \item \texttt{gx}: open URL / http under cursor.
    \end{itemize}
\end{itemize}
%
%
\end{document}
